\documentclass[lettersize,journal]{IEEEtran}
\usepackage{amsmath,amsfonts}
\usepackage{algorithmic}
\usepackage{algorithm}
\usepackage{array}
\usepackage[caption=false,font=normalsize,labelfont=sf,textfont=sf]{subfig}
\usepackage{textcomp}
\usepackage{stfloats}
\usepackage{url}
\usepackage{verbatim}
\usepackage{graphicx}
\usepackage{cite}
% updated with editorial comments 8/9/2021

\begin{document}

\title{MAE 298 Introduction to PDEs \\ Electrochemical Modeling of Batteries}

\author{Jonathan Dorsey}


% Remember, if you use this you must call \IEEEpubidadjcol in the second
% column for its text to clear the IEEEpubid mark.

\maketitle

\begin{abstract}
This
\end{abstract}

\section{Introduction}
\IEEEPARstart{P}{artial} differential equations (PDEs) provide a the mathematical basis for describing a significant number of scientific pheonimon; however, while the study of PDEs is import, it can often be more challenging to develope PDE which models these phenomina from first principles than it is to understand the structure of a given PDE. This can be more readily seen when investigating more complex systems which couple several PDEs together to describe the full dynamics of the system. To this point, electrochemical models of lithium-ion batteries fall into this category of complex systems. Even simple electrochemical models of batteries consist of anywhere upto five PDEs. The objective of this paper will be to derive the salient PDEs describing the electrochemical battery model known as Doyle-Fuller-Newman (DFN) model, with enough context and background about the first principles being applied to enlighten the reader as to how and why the given PDE model is meaningful.

\section{Doyle-Fuller-Newman Model Overview}
The DFN model has long been held as the gold standard electrochemical battery model. This model describes at the microscale dynamics of a planar battery cell (SHOWN IN FIG) which has been volume averaged to account for usage of this model as a full battery cell, which accounts for behaviors such as mass and charge conservation, as well as the process of intercalation and deintercalation where lithium is distributing through the material of electrode material or coating during charging and discharging respectively. This model can be broken down in to the five following categories. In turn, this paper will investigate each category, and provide the context and first principles analysis which yield appropriate governing PDEs.

\begin{enumerate}
  \item
  \item
  \item
  \item

\end{enumerate}






\subsection{Charge Conservation in Homogeneous Solid}

The first step in this model is to use the principle of conservation of charge, in the solid phase of the battery. This means that charge is not created or destroyed within the battery. The principle used to derive this equation is the point for of \textbf{Ohms Law}. This means that we assume that through the electrode material, electron movement is caused by drifting of charge, as perscribed by Ohms Law. Namely...

\[
\textbf{i} = \sigma\textbf{E}
\]

Where \textbf{i} is the current density [$Am^{-2}$], $\sigma$ is the conductivity of the electrode material, and \textbf{E} is the applied electric field. This is an alternate form of Ohms Law ($V = IR$)


\subsection{Mass Conservation in Homogeneous Solid}
\subsection{Charge Conservation in Homogeneous Electrolyte}
\subsection{Mass Conservation in Homogeneous Electrolyte}
\subsection{Lithium Movement Between Solid \& Electrolyte Phases}




\section{Derivation}

\section{Analysis of Governing PDEs}

\section{Numerical Solution}

\section{Results}

\section{Conclusion}
The conclusion goes here.


\section*{Acknowledgments}
This should be a simple paragraph before the References to thank those individuals and institutions who have supported your work on this article.

\section{References Section}
You can use a bibliography generated by BibTeX as a .bbl file.
 BibTeX documentation can be easily obtained at:
 http://mirror.ctan.org/biblio/bibtex/contrib/doc/
 The IEEEtran BibTeX style support page is:
 http://www.michaelshell.org/tex/ieeetran/bibtex/

 % argument is your BibTeX string definitions and bibliography database(s)
%\bibliography{IEEEabrv,../bib/paper}
%
\section{Simple References}
You can manually copy in the resultant .bbl file and set second argument of $\backslash${\tt{begin}} to the number of references
 (used to reserve space for the reference number labels box).

\begin{thebibliography}{1}
\bibliographystyle{IEEEtran}

\bibitem{ref1}
{\it{Mathematics Into Type}}. American Mathematical Society. [Online]. Available: https://www.ams.org/arc/styleguide/mit-2.pdf

\bibitem{ref2}
T. W. Chaundy, P. R. Barrett and C. Batey, {\it{The Printing of Mathematics}}. London, U.K., Oxford Univ. Press, 1954.

\bibitem{ref3}
F. Mittelbach and M. Goossens, {\it{The \LaTeX Companion}}, 2nd ed. Boston, MA, USA: Pearson, 2004.

\end{thebibliography}


\newpage

\section{Biography Section}

\vspace{11pt}

% \bf{If you include a photo:}\vspace{-33pt}
% \begin{IEEEbiography}[{\includegraphics[width=1in,height=1.25in,clip,keepaspectratio]{fig1}}]{Michael Shell}
% Use $\backslash${\tt{begin\{IEEEbiography\}}} and then for the 1st argument use $\backslash${\tt{includegraphics}} to declare and link the author photo.
% Use the author name as the 3rd argument followed by the biography text.
% \end{IEEEbiography}

\vspace{11pt}

\vfill

\end{document}
